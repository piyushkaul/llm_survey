%%%%%%%%%%%%%%%%%%%%%%%%%%%%%%%%%%%%%%%%%
% Beamer Presentation
% LaTeX Template
% Version 1.0 (10/11/12)
%
% This template has been downloaded from:
% http://www.LaTeXTemplates.com
%
% License:
% CC BY-NC-SA 3.0 (http://creativecommons.org/licenses/by-nc-sa/3.0/)
%
%%%%%%%%%%%%%%%%%%%%%%%%%%%%%%%%%%%%%%%%%

%----------------------------------------------------------------------------------------
%	PACKAGES AND THEMES
%----------------------------------------------------------------------------------------

\documentclass{beamer}

\mode<presentation> {

% The Beamer class comes with a number of default slide themes
% which change the colors and layouts of slides. Below this is a list
% of all the themes, uncomment each in turn to see what they look like.

%\usetheme{default}
%\usetheme{AnnArbor}
%\usetheme{Antibes}
%\usetheme{Bergen}
%\usetheme{Berkeley}
%\usetheme{Berlin}
%\usetheme{Boadilla}
%\usetheme{CambridgeUS}
%\usetheme{Copenhagen}
%\usetheme{Darmstadt}
%\usetheme{Dresden}
%\usetheme{Frankfurt}
%\usetheme{Goettingen}
%\usetheme{Hannover}
%\usetheme{Ilmenau}
%\usetheme{JuanLesPins}
%\usetheme{Luebeck}
\usetheme{Madrid}
%\usetheme{Malmoe}
%\usetheme{Marburg}
%\usetheme{Montpellier}
%\usetheme{PaloAlto}
%\usetheme{Pittsburgh}
%\usetheme{Rochester}
%\usetheme{Singapore}
%\usetheme{Szeged}
%\usetheme{Warsaw}

% As well as themes, the Beamer class has a number of color themes
% for any slide theme. Uncomment each of these in turn to see how it
% changes the colors of your current slide theme.

%\usecolortheme{albatross}
%\usecolortheme{beaver}
%\usecolortheme{beetle}
%\usecolortheme{crane}
%\usecolortheme{dolphin}
%\usecolortheme{dove}
%\usecolortheme{fly}
%\usecolortheme{lily}
%\usecolortheme{orchid}
%\usecolortheme{rose}
%\usecolortheme{seagull}
%\usecolortheme{seahorse}
%\usecolortheme{whale}
%\usecolortheme{wolverine}

%\setbeamertemplate{footline} % To remove the footer line in all slides uncomment this line
%\setbeamertemplate{footline}[page number] % To replace the footer line in all slides with a simple slide count uncomment this line

%\setbeamertemplate{navigation symbols}{} % To remove the navigation symbols from the bottom of all slides uncomment this line
}

\usepackage{graphicx} % Allows including images
\usepackage{booktabs} % Allows the use of \toprule, \midrule and \bottomrule in tables
%\usepackage{movie15}
%%%%% SPECIAL PACKAGES %%%%%%%%%%
\usepackage{setspace}
%\usepackage[ngerman]{babel}
\usepackage[utf8]{inputenc}
\usepackage{fancyhdr}
\usepackage{tabularx}
%\renewcommand{\rmdefault}{phv}
%\renewcommand{\sfdefault}{phv}

\setcounter{tocdepth}{2} % to get subsubsections in toc 
% cf. http://www.latex-community.org/forum/viewtopic.php?f=47&p=44760

\usepackage{amssymb,latexsym}
\usepackage{amsmath, amsthm}

%for bibliography; installation using 'sudo tlmgr install amsrefs'
\usepackage{amsrefs}

\usepackage{graphics}
\usepackage{animate}
%\usepackage{xmpmulti}

\usepackage{hyperref}
\hypersetup{colorlinks=true, urlcolor=blue}

\usepackage{cancel} % http://jansoehlke.com/2010/06/strikethrough-in-latex/

\usepackage{listings} % http://en.wikibooks.org/wiki/LaTeX/Source_Code_Listings
% http://olmjo.com/files/teaching/PSC505/LaTeXandR.pdf

% package for flower symbol (\ding(96))
\usepackage{pifont}
% required installation: sudo apt-get install texlive-fonts-recommended (30MB)
% http://tug.ctan.org/info/symbols/comprehensive/symbols-a4.pdf

\usepackage{tikz} % for diagrams
\usetikzlibrary{matrix,positioning,arrows,calc,decorations.pathmorphing,shapes}
% for snaky lines (http://tex.stackexchange.com/questions/209942/curved-arrows-in-tikz) 
\tikzset{snake it/.style={-stealth,
decoration={snake, 
    amplitude = .4mm,
    segment length = 2mm,
    post length=0.9mm},decorate}}

\usepackage[parfill]{parskip}

\usepackage{framed} %for putting some text in boxes using \begin{framed}

\usepackage{enumerate}

%for displaying tensor indices properly. requires installation of tensor package using 'sudo tlmgr install tensor'
\usepackage{tensor}

%for placing captions of figures on the side instead of above/below the figure
\usepackage{sidecap}
\linespread{1.2}

%plain makes sure that we have page numbers
%\pagestyle{plain}

\theoremstyle{plain}
\newtheorem{axiom}{Axiom}
\newtheorem*{main}{Main Theorem}
\newtheorem{proposition}{Proposition}

\theoremstyle{definition}

\theoremstyle{remark}
\newtheorem*{notation}{Notation}

\numberwithin{equation}{section}
\numberwithin{figure}{section}
\numberwithin{theorem}{section}

%symbol for maps
\renewcommand{\to}{\longrightarrow}
\newcommand{\injmapto}{\hookrightarrow}
\newcommand{\surjmapto}{\twoheadrightarrow}
\newcommand{\linearmapto}{\stackrel{\sim}{\longrightarrow}}
\newcommand{\projmapto}{\stackrel{\pi}{\longrightarrow}}

%for real numbers
\newcommand{\R}{\mathbb{R}}

% manifold, atlas and topology
\newcommand{\A}{\mathcal{A}}
%\newcommand{\O}{\mathcal{O}}
\newcommand{\mfd}{(M, \mathcal{O}, \mathcal{A})}

\newcommand{\after}{\circ}
\newcommand{\stdtop}{\mathcal{O}_{std}}
\newcommand{\cibasis}[2][]{\frac{\partial #1}{\partial #2}}

%connection coefficient functions or gammas
\newcommand{\ccf}[2]{\Gamma\indices{^{#1}_{#2}}}
\newcommand{\ccfx}[3]{\left(\Gamma_{#3}\right)\indices{^{#1}_{#2}}} % with chart index

%set theory symbols
%\renewcommand{\exists}{\exists\,}
%\renewcommand{\forall}{\forall\,}

%This defines a new command \questionhead which takes one argument and prints out Question #. with some space.
\newcommand{\questionhead}[1]
  {
   \noindent{\small\bf Question #1.}
  }

\newcommand{\problemhead}[1]
  {
   \noindent{\small\bf Problem #1.}
  }

\newcommand{\exercisehead}[1]
  { \smallskip
   \noindent{\small\bf Exercise #1.}
  }

\newcommand{\solutionhead}[1]
  {
   \noindent{\small\bf Solution #1.}
  }

\newcommand{\bubblethis}[2]{
  \tikz[remember picture,baseline]{\node[anchor=base,inner sep=0,outer sep=0](#1) {#1};\node[overlay,cloud callout,callout relative pointer={(0.2cm,-0.7cm)}, aspect=2.5,fill=white!90] at ($(#1.north)+(-0.5cm,1.6cm)$) {#2};}
}



%%%%%%%%%%%%%%%%%%%%%%%%%%%%%


%----------------------------------------------------------------------------------------
%	TITLE PAGE
%----------------------------------------------------------------------------------------
\title[Topological Manifolds]{Topological Manifolds} % The short title appears at the bottom of every slide, the full title is only on the title page

\author{Piyush Kaul} % Your name
\institute[IITD] % Your institution as it will appear on the bottom of every slide, may be shorthand to save space
{
EE Dept. (IIT-D) \\ % Your institution for the title page
\medskip
\textit{piyushkaul@gmail.org} % Your email address
}
\date{\today} % Date, can be changed to a custom date

\begin{document}

\begin{frame}
\titlepage % Print the title page as the first slide
\end{frame}

\begin{frame}
\frametitle{Overview} % Table of contents slide, comment this block out to remove it
\tableofcontents % Throughout your presentation, if you choose to use \section{} and \subsection{} commands, these will automatically be printed on this slide as an overview of your presentation
\end{frame}

%----------------------------------------------------------------------------------------
%	PRESENTATION SLIDES
%----------------------------------------------------------------------------------------

%------------------------------------------------
\section{Topology} % Sections can be created in order to organize your presentation into discrete blocks, all sections and subsections are automatically printed in the table of contents as an overview of the talk
%------------------------------------------------


\begin{frame}
\frametitle{Topology}
\begin{definition}
  Let $M$ be a set and $\mathcal{P}(M)$ be the power set of $M$, i.e., the set of all subsets of $M$.   \\
A set $\mathcal{O} \subseteq \mathcal{P}(M)$ is called a \textbf{topology}, if it satisfies the following:
\begin{enumerate}
  \item[(i)] $\emptyset \in \mathcal{O}$, $M \in \mathcal{O}$ 
\item[(ii)] $U \in \mathcal{O}$, \, $V \in \mathcal{O} \implies U \cap V \in \mathcal{O}$ 
\item[(iii)] $U_{\alpha} \in \mathcal{O}$, \, $\alpha \in \mathcal{A}$ ($\mathcal{A} \text{ is an index set )} \implies \left( \bigcup_{\alpha \in \mathcal{A}} U_{\alpha} \right) \in \mathcal{O}$
\end{enumerate}
\end{definition}

\end{frame}

\begin{frame}
\frametitle{Open and Close Sets}
\textbf{Terminology}:
\begin{enumerate}
\item the tuple $(M , \mathcal{O})$ is a \textbf{topological space}.
\item $\mathcal{U} \in M$ is an \textbf{open set} if $\mathcal{U} \in \mathcal{O}$.
\item $\mathcal{U} \in M$ is a \textbf{closed set} if $M \setminus \mathcal{U} \in \mathcal{O}$.
\end{enumerate}

\begin{definition}
  $(M , \mathcal{O})$, where $\mathcal{O} = \lbrace \emptyset, M\rbrace$ is called the \textbf{chaotic topology}.
\end{definition}

\begin{definition}
  $(M , \mathcal{O})$, where $\mathcal{O} = \mathcal{P}(M)$ is called the \textbf{discrete topology}.
\end{definition}
\end{frame}


\begin{frame}
\frametitle{Topology Examples}
\begin{figure}
%\includegraphics[width=5cm]{topology_examples}
\end{figure}
\end{frame}



\section{Parallel Transport}
\section{Lie Groups and Algebras}


%------------------------------------------------

\begin{frame}
\frametitle{References}
\footnotesize{
\begin{thebibliography}{99} % Beamer does not support BibTeX so references must be inserted manually as below
\bibitem[Schuller, 2012]{p1}  Fredric Schuller(2012)
\textit{Lectures on Geometrical Anatomy of Physics}
%\verb{https://www.youtube.com/watch?v=V49i_LM8B0E}

\bibitem[Isham, 1989]{p2} Chris Isham(1989)
\textit{ Modern Differential Geometry for Physicists}

\bibitem[Nash, 1989]{p3} Charles Nash(2011)
\textit{ Topology and Geometry for Physics }

\bibitem[Nakahara, 2003]{p4} Mikio Nakahara(2003)
\textit{ Geometry, Topology and Physics}

\end{thebibliography}
}
\end{frame}

%------------------------------------------------

\begin{frame}
\Huge{\centerline{The End}}
\end{frame}

%----------------------------------------------------------------------------------------

\end{document} 